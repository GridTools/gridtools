\documentclass[12pt]{article}
\usepackage{graphicx}
\usepackage{amsmath,amssymb,bm}
\usepackage{listings}
\usepackage[scale=0.75]{geometry}

\lstset{language=C++,basicstyle=\footnotesize\ttfamily,breaklines=true}

\setlength\parindent{0pt}
\setlength{\parskip}{1em}

\begin{document}
  Chapter 3 of Wan, 2009 \cite{Wan} gives a very detailed explanation on the numerics in the dycore of ICON. These further examples of operators and discretization methods provide more to think about on the expressiveness of the GridTools.

  \section{More evidence calling for fields on multiple location}
  \subsection{Examples}
  We start from these edge-to-cell and cell-to-edge averaging operators that are used a lot in discretization defined as (the following numbering of equations refers to that in \cite{Wan})
  \begin{equation}
      \bar{\varrho}_i=\sum\limits_{l \in \mathcal{E}(i)}\frac{A_{i,l}}{A_i}\varrho_{l} \tag{3.13}
  \end{equation}
  \begin{equation}
      \breve{\varrho}_l=\sum\limits_{j=1,2}\frac{A_{i(l,j)}}{A_l}\varrho_{i(l,j)} \tag{3.14}
  \end{equation}
  Here $i(l,1)$ and $i(l,2)$ denote the neighboring cells to edge $l$. $A_l$ and $A_i$ is the area associated to edge $l$ and cell $i$. As $A_i$ and $A_l$ vary on the grid, this cannot be considered as a simple arithmetic average. Examples of (3.13) and (3.14) are
  \begin{equation}
    \text{div}(\bm{v}\Delta p)_{i, k}=\frac{1}{A_i} \sum\limits_{l\in\mathcal{E}(i)}v_{n_l,k}\Delta\breve{p}_{l,k}(\bm{N}_l\cdot\bm{n}_{i,l})l
    \tag{3.18}
  \end{equation}
  \begin{equation}
    \overline{K}_{i,k}=\sum\limits_{l\in\mathcal{E}(i)} \frac{A_{i,l}}{A_i}v^2_{n_l,k}
    \tag{3.25}
  \end{equation}
  \begin{equation}
    \text{div}(\bm{v}T\Delta p)_{i, k}=\frac{1}{A_i} \sum\limits_{l\in\mathcal{E}(i)}v_{n_l,k}\breve{T}_{l,k}\Delta\breve{p}_{l,k}(\bm{N}_l\cdot\bm{n}_{i,l})l
    \tag{3.18}
  \end{equation}

  Basically, we have such a form
  \[\text{operator}(\varrho)_i = \sum\limits_{l \in \mathcal{E}(i)}w_{i, l}\varrho_{l}\]
  \[\text{operator}(\varrho)_l = \sum\limits_{j=1,2}w_{l, i(l,j)}\varrho_{i(l,j)}\]
  This more general form can also describe
  \begin{equation}
    \overline{\overline\psi}_l = \frac{A_{i(i,2),l}}{A_l}\psi_{i(l,1)} + \frac{A_{i(i,1),l}}{A_l}\psi_{i(l,2)}
    \tag{3.51}
  \end{equation}
  A more complicated example also fits this form, considering $w_{i, l}=\frac{1}{A_i \Delta p_{i, k}}2A_{i,l}$:
  \begin{equation}
      \Big(\frac{R_dT}{p}\bm{v}\cdot\nabla p\Big)_{i, k} = \frac{1}{A_i \Delta p_{i, k}} \sum\limits_{l\in\mathcal{E}(i)}\bigg[\Big(\frac{R_dT}{p}\nabla p\Big)_{l, k}\cdot \bm{N}_l\bigg] u_{l,k} \Delta\breve{p}_{l,k}2A_{i,l}
      \tag{3.57}
    \end{equation}
  The average divergence operator is a cell-to-cell averaging, which has the form
  \[\text{operator}(\varrho)_i = \sum\limits_{j \in \{i\}\cup\mathcal{N}(i)}w_{i, j}\varrho_{j}\]
  where $\mathcal{N}_i$ denote the neighboring cells to cell $i$.

  \subsection{Problems}
  These operators are beyond what a uniform sign or different strategies regarding coloring can describe.

  \section{Proposal}
  A proposed syntax to describe, for example,
  \[\text{operator}(\varrho)_i = \sum\limits_{l \in \mathcal{E}(i)}w_{i, l}\varrho_{l}\]
  would be
  \begin{lstlisting}[frame=single]
struct edge_to_cell_averaging_functor {
  typedef in_accessor<0, icosahedral_topology_t::edges, extent<1> > rho;
  typedef in_accessor<1, icosahedral_topology_t::cells, icosahedral_topology_t::edges, extent<1> > weights;
  typedef inout_accessor<2, icosahedral_topology_t::cells> out_cells;
  typedef boost::mpl::vector<rho, weights, out_cells, cell_area> arg_list;

  template<typename Evaluation>
  GT_FUNCTION static void Do(Evaluation const &eval, x_interval)
  {
      auto ff = [](const double _in1, const double _in2, const double _res) -> double { return _in1 * _in2 + _res; };

      eval(out_cells()) = eval(on_edges(ff, 0.0, eval(weights), rho()));
  }
};

auto weights = icosahedral_grid.make_storage<icosahedral_topology_t::cells, icosahedral_topology_t::edges, double >("weights");
  \end{lstlisting}
Note that here \texttt{weights} is a field on both \texttt{cells} and \texttt{edges}. The idea is that by calling \texttt{eval(weights)} we get an edge storage regarding current cell. This storage is virtual in the sense that only edges on that cell is stored, other entries are all empty.



  \begin{thebibliography}{9}
  \bibitem{Wan}
  Wan, Hui. \textit{Developing and testing a hydrostatic atmospheric dynamical core on triangular grids}. Diss. University of Hamburg Hamburg, 2009.
  \end{thebibliography}

\end{document}
